\begin{frame}
    \frametitle{Introducción}
    \framesubtitle{¿Qué es FMM?}

    \begin{center}
        Es un algoritmo TreeCode que utiliza dos representaciones del campo gravitatorio,
        campos lejanos (multipole) y expansiones locales.
    \end{center}
\end{frame}

\begin{frame}
    \frametitle{Introducción}
    \framesubtitle{¿Por qué es importante?}

    \begin{itemize}
        \item Cálculo muy rápido del potencial (campo).
        \item Más fácil computacionalmente trabajar con el potencial que con la fuerza.
        \begin{itemize}
            \item La fuerza es un vector. $$F_{ij} =G \cdot \frac{m_i \cdot m_j}{||r_{ij}||^{2}} \cdot \frac{r_{ij}}{||r_ij||}$$
            \item El potencial es un escalar. $$\Phi_{i} = \sum_{j=0}^{N} \frac{m_{j}}{r}$$
        \end{itemize}
        \item $F = - \nabla \Phi $
    \end{itemize}
\end{frame}


\begin{frame}
    \frametitle{Introducción}
    \framesubtitle{¿Cuál es la idea?}

    \begin{itemize}
        \item La estrategia es calcular una expresión compacta para el potencial.
        \begin{itemize}
            \item fácil de evaluar con sus derivadas.
        \end{itemize}
        \item Esto se consigue evaluando el potencial como un ``desarrollo multipolar'' (multipole expansion)
        \begin{itemize}
            \item Toma la idea de una expansión de Taylor.
            \item Es exacta cuando el valor de $r^2$ es grande.
                $$r^{2} = x^{2} + y^{2} + z^{2}$$
                  Siendo $r$ la distancia entre dos cuerpos.
        \end{itemize}
    \end{itemize}
\end{frame}

\begin{frame}
    \frametitle{Introducción}
    \framesubtitle{¿En qué se diferencia con un TreeCode?}

    \begin{itemize}
        \item<1-> FMM calcula una expresión para el \blue{potencial} en todos los puntos, no la \blue{fuerza} como lo hace el TC.
        \item<2-> FMM usa \blue{más información} que sólo la masa y el centro de las partículas en una caja.
        \item<2-> Esta expansión compleja es \blue{más precisa}, pero también \red{más cara} si se usan expansiones de mayor orden.
        \item<3-> Decidir si ocupamos la masa y el centro de masa de una caja de lado $n$ en un punto distante,
                  o si tenemos que descomponerla, depende sólo de la \blue{ubicación y tamaño de la caja},
                  \textbf{no} de la \blue{ubicación del centro de masa} de la caja.
    \end{itemize}
\end{frame}
