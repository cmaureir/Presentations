\begin{frame}
    \frametitle{Introducción}
    \framesubtitle{¿Qué es FMM?}

    \begin{center}
        Es un algoritmo TreeCode que utiliza dos representaciones del campo gravitatorio,
        campos lejanos (multipole) y expansiones locales.
    \end{center}
\end{frame}

\begin{frame}
    \frametitle{Introducción}
    \framesubtitle{¿Por qué es importante?}

    \begin{itemize}
        \item Cálculo muy rápido del potencial (campo).
        \item Más fácil computacionalmente trabajar con el potencial que con la fuerza.
        \begin{itemize}
            \item La fuerza es un vector. $$F_{ij} =G \cdot \frac{m_i \cdot m_j}{||r_{ij}||^{2}} \cdot \frac{r_{ij}}{||r_ij||}$$
            \item El potencial es un escalar. $$\Phi_{i} = \sum_{j=0}^{N} \frac{m_{j}}{r}$$
        \end{itemize}
        \item $F = - \nabla \Phi $
    \end{itemize}
\end{frame}


\begin{frame}
    \frametitle{Introducción}
    \framesubtitle{¿Cuál es la idea?}

    \begin{itemize}
        \item La estrategia es calcular una expresión compacta para el potencial.
        \begin{itemize}
            \item fácil de evaluar con sus derivadas.
        \end{itemize}
        \item Esto se consigue evaluando el potencial como un ``desarrollo multipolar'' (multipole expansion)
        \begin{itemize}
            \item Toma la idea de una expansión de Taylor.
            \item Es exacta cuando el valor de $r^2$ es grande.
                $$r^{2} = x^{2} + y^{2} + z^{2}$$
        \end{itemize}
    \end{itemize}
\end{frame}

\begin{frame}
    \frametitle{Introducción}
    \framesubtitle{¿En qué se diferencia con un TreeCode?}

    \begin{itemize}
        \item<1-> FFM calcula una expresión para el \blue{potencial} en todos los puntos, no la \blue{fuerza} como lo hace el TC.
        \item<2-> FFM usa \blue{más información} que sólo la masa y el centro de las partículas en una caja.
                  Esta expansión mas complicada es más precisa, pero también más cara si se usan expansiones de mayor orden.
        \item<3-> La decisión de usar la masa y el centro de masa de una caja de longitud $n$ en un punto distante,
                  o si $n$ debe ser descompuesto en sus hijos, depende sólo de la \blue{posición y tamaño de la caja},
                  \textbf{no} de la \blue{ubicación del centro de masa} de la caja.
    \end{itemize}
\end{frame}


\begin{frame}
    \frametitle{Introducción}
    \framesubtitle{¿Cuándo conviene y cuándo no conviene usarlo?}

    \begin{itemize}
        \item<1-> Este algoritmo no es adecuado para estudios de sistemas de colisiones: (McMillan y S. J. Aarseth (1993))
        \begin{itemize}
            \item Es lento comparado con otros métodos.
            \item Deriva su tiempo de CPU de orden O(1) por cada evaluación de la fuerza, para determinar las fuerzas en un número arbitrario de partículas en tiempo casi constante.
        \end{itemize}
        \item<2-> El algoritmo es muy adecuado para aplicaciones en las que todas las partículas tienen los intervalos de tiempo iguales o similares.
    \end{itemize}
\end{frame}

