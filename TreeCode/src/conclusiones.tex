\frame
{
\frametitle{Conclusiones}
\begin{itemize}
	\item<1-> Muestra \green{nuevo} enfoque para abordar el problema.
	\item<2-> Es más \green{rápido}, pues no recorre todos los cuerpos.
	\item<3-> Disminuye el \blue{orden} del algoritmo de $O(n^2)$ a $O(n \log n)$.
	\item<4-> Generamos \red{errores}, lo que nos lleva a tener una menor precisión.
	\begin{itemize}
		\item<5-> Podemos controlar la precisión, modificando $\theta$.
		\item<6-> Desarrollo multipolar (\emph{Multipole expansions}) lo corrige!.  
	\end{itemize}
\end{itemize}
}
