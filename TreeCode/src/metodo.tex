\frame
{
\frametitle{Barnes-Hut Tree Code}
\framesubtitle{Idea}

\begin{center}
``Agrupar cuerpos ceranos y aproximarlos en un solo cuerpo''
\end{center}

\begin{itemize}
	\item Si el grupo está muy lejos, se aproximan los efectos gravitacionales
		 utilizando el centro de masa.
	\item Centro de masa, promedio de las posiciones de un cuerpo en un grupo,
		 ponderado por la masa.
\end{itemize}
}

\frame
{
\frametitle{Barnes-Hut Tree Code}
\framesubtitle{Idea}

\blue{Centro de Masa}


Formalmente si dos cuerpos con posiciones (x1,y1), (x2,y2) y masas m1, m2,
la masa total del centro de masa (x,y) está dada por:

\begin{eqnarray}
	m &=& m_{1} + m_{2} \nonumber \\
	x &=& (x_{1}\cdot m_{1} + x_{2}\cdot m_{2}) / m \nonumber \\
	y &=& (y_{1}\cdot m_{1} + y_{2}\cdot m_{2}) / m \nonumber \\
\end{eqnarray}

}

\frame
{
\frametitle{Barnes-Hut Tree Code}
\framesubtitle{Idea}

El algoritmo es un esquema inteligente para agrupar cuerpos
suficientemente cerca.
Recursivamente divide el conjunto de cuerpos en grupos
guardandolos en un quad-tree.
Un quad-tree es similar a un binary-tree, excepto que cada nodo
tiene 4 hijos (algunos pueden estar vacios),.
Cada nodo representa una region de dos espacios dimensiones.

El nodo más alto representa todo el espacio,
y sus cuatro hijos representan cuatro cuadrantes en el espacio,
}

\frame
{
\frametitle{Barnes-Hut Tree Code}
\framesubtitle{Aproximación}

Como se muestra en  diagram1, el espacio es recursivamente
subdividido en cuadrantes
hasta que cada subdivisión contenta 0 o 1 cuerpo,
(algunas regiones no tienen cuerpos en todos los cuadrantes.

}

\frame
{
\frametitle{Barnes-Hut Tree Code}
\framesubtitle{Aproximación}

Cada nodo externo, representa un único cuerpo.
Cada nodo interno representa el grupo de cuerpos debajo de ella,
y guarda el centro de masa y el total de masa de todos los
cuerpos hijos.

Diagram2
}


\frame
{
\frametitle{Barnes-Hut Tree Code}
\framesubtitle{Cálculo de la fuerza}

Para calcular la fuerza de la red en un particular cuerpo,
recorre los nodos del arbol, partiendo de la raíz.
Si el centro de masa de un nodo interno es suficientemente lejos
del cuerpo, aproxima los cuerpos contenidos
en esa parte del arbol en un solo cuerpo cuya posiciń es
el centro de masa del grupo y cuya masa es el total de la masa del
grupo.
El algoritmo es rápido ya que no necesita examinar individualmente
cualquier cuerpo del grupo.

Si el nodo interno no está lo suficientemente lejos del cuerpo,
recursivamente hay que recorrer cada uno de los sub-árboles.
Para determinar si un nodo es suficientemente lejos, calcula
el cuociente s/d, donde s es el ancho de la región representada
por el nodo interno, y d es la distancia entre
el cuerpo y el centro de masa del nodo.
Despues, compara ese ratio con un valor umbral \theta.
If s/d < \theta, entonces el nodo interno esta lo suficiente lejos.
Ajustando el parametro \theta, podemos cambiar la velocidad
y la presición de la simulacion.
Se suele utilizar \theta = 0.5.
Notar que si \theta = 0, entonces ningún nodo interno
es tratado como un sólo cuerpo, y el algoritmo
degenera la fuerza bruta.
}

\frame
{
\frametitle{Barnes-Hut Tree Code}
\framesubtitle{Construyendo un árbol}

Construyendo el Árbol Barnes-Hut

Para construir el árbol, inserta los cuerpos uno despues de otro.
Para insertar un cuerpo b en el arbol con raíz en el nodo x,
usar los siguientes pasos recursivos:

1. Si el nodo x no contiene un cuerpo, poner el nuevo cuerpo b ahí.
2. Si el nodo x es un nod interno, actualizar el centro de masa
   y el total de la masa de x. Recursivamente inserta el cuerpo b
   en el cuadrante apropiado.
3. Si el nodo x es un nodo externo, digamos que contiejne un cuerpo
   llamado c, entonces hay dos cuerpos, b y c en la misma región.
   Subdivide la región creando cuatro hijos. Entonces, recursivamente
   inserta ambos cuerpos, b y c, en el cuadrante apropiado. Puesto
   que b y c todavía pueden terminar en el mismo cuadrantes, podrian
   haber varias subdivisiones durante una inserción simple.
   Finalmente, actualizar el centro de masa y la masa total de x.

Como ejemplo, consideramos los siguientes 5 cuerpos en el diagrama
a continuación. En nuestros ejemplos, usamos la convención que las
ramas, desde izquierda a derecha, representan los cuadrantes
NO, NE, SO y SE, respectivamente.
El arbol va a traves de las siguientes etapas a medida que los
cuerpos son insertados.

Ejemplo con diagramas.

El nodo raíz contiene el centro de masa y el total 
de masa de los 5 cuerpos. Los otros dos nodos internos
contienen entre ellos el centro de masa y la masa total
de los cuerpos b,c y d.
}

\frame
{
\frametitle{Barnes-Hut Tree Code}
\framesubtitle{Cálculo de la fuerza sobre un cuerpo}

Calculando la fuerza que actua sobre un cuerpo.

Para calcular la fuerza que ejerce la red sobre un cuerpo b,
usar el siguiente procedimiento recursivo, comenzando con
la raiz del quad-tree:

1. Si el nodo actual es un nodo externo (y no es el cuerpo b),
   calcular la fuerza ejercida por el nodo actual sobre b,
   y agregar esta cantidad a la fuerza de la red sobre b.
2. De otro modo, calcular el ratio s/d. Si s/d < \theta,
   tratar este nodo interno como un unico cuerpo, y calcular
   la fuerza que ejerce sobre el cuerpo b, y agrega esta fuerza
   a la cantidad de fuerza de la red sobre b.
3. De otro modo, correr el procedimiento recursivamente en cada
   uno de los actuales nodos hijos.

Como ejemplo, para calcular la fuerza de la red sobre el cuerpo a
comenzaremos en la raiz, la cual es un nodo interno.
Esta representa el centro de masa de los 6 cuerpos
a, b, c, d, e y f, los cuales tienen masas 1, 2, 3, 4, 5 y 6 Kg,
respectivamente.

El calculo de fuerza procede de esta forma:

1. El primer nodo examinado es la raiz. Comparando el cuerpo A
   a el nodo centro de masa (punto blanco), el cuociente
   s/d = 100/43.1 > \theta = 0.5, por lo que realizamos el proceso
   recursivo en cada hijo de la raiz.

2. El primer hijo es el cuerpo A, por lo que no ejerce fuerza sobre
   si mismo, no hacemos nada.

3. Este hijo representa el cuadrante NE del espacio, y contiene
   el centro de masa de los cuerpos b, c, d y e.
   Ahora s/d = 50/62.7 > \theta por lo que recursivamente
   calculamos la fuerza ejercida por el primer nodo hijo no vacio.

4. Este es tambien un nodo interno, representando el cuadrante
   NE del padre, y conteniendo el centro de masa de los cuerpos
   b, c y d.
   Ahora s/d = 25/66.9 < \theta.
   Tratando el nodo interno como un solo cuerpo, cuya masa
   es la suma de las masas de b, c y d, calculamos la fuerza
   ejercida sobre el cuerpo a, y se agrega al valor
   de la fuerza ejercida de la red sobre a.
   Dado que el padre de este nodo no posee mas hijos,
   continuamos examinando los otros hijos de la raiz.

5. El siguiente hijo es el que contiene el cuerpo e.
   Este es un nodo externo, por lo que calculamos la fuerza
   entre los cuerpos a y e, y los agregamos a la fuerza
   ejercida de la red sobre a.

6- Habiendo examinado todos los hermanos en este nivel,
   nos movemos al siguiente hijo del padre, lo que nos lleva
   al nodo que contiene el cuerpo f.
   Debido a que es un nodo externo calculamos la fuerza
   y la agregamos a la fuerza ejercida de la red sobre a.
}


%Tree Codes are gridless, have no preferred geometry and can incorporate
%either vacuum or periodic boundary conditions.
%In addition, they waste no time simulating regions devoid of matter.
%Hence, Tree Codes are particularly effective for modeling collisions
%between galaxies.
%
%Forces on all particles are obtained with O(N log N) operations.
%The down side is that tree codes require a large amount of auxillary storage.
%
%Note: The standard Barnes Tree Code is not as accurate, however,
%as the Fast Multipole Method (FMM), to be discussed later.
%The Barnes-Hut type algorithms are essentially dumb FMM algorithms with
%order-0 multipole expansions.
%FMM would only be slower than the Barnes Hut if you want more accuracy
%(but for a given accuracy, FMM is likely faster).
%
%McMillan and Aarseth (1993) remedy the limited accuracy shortcoming by
%using a higher order integration scheme, along with block time steps and
%multipole expansions.
%The tree is not reconstructed at every step, but is instead predicted along
%with the particles with indifidual nodes rebuilt as needed.
